\documentclass[a4paper]{article}
\usepackage[14pt]{extsizes}
\usepackage[T2A]{fontenc}
\usepackage[utf8]{inputenc}
\usepackage[english, russian]{babel}
\usepackage{geometry}
\geometry{left=2cm}
\geometry{right=1.5cm}
\geometry{top=1.5cm}
\geometry{bottom=1.5cm}
\usepackage{hyperref}
\usepackage{graphicx} 
\usepackage{tabto}
\usepackage{setspace}
\usepackage{color} 
\usepackage{listings}
\usepackage{algorithm}
\usepackage{algpseudocode}

\begin{document}
\begin{onehalfspacing}
\begin{titlepage}
	\begin{center}
		{Федеральное агенство связи \\ Федеральное государственное бюджетное образовательное учреждение высшего образования "Сибирский государственный университет телекоммуникаций и информатики"\\[5pt]}
		\vspace{1.5cm}
		\begin{flushright}
			{\normalsize\bfseries Кафедра ТСиВС}\\
		\end{flushright}
		\vspace{2cm}
		{\Large Отчет по курсовой работе\\по дисциплине: <<Моделирование распределенных систем>>\\на тему: <<Моделирование движения абонентов по дорогам>> } \\[2cm]
		\begin{flushleft}
			{\tab Выполнили:\\\tab студенки группы ИА-831:\\\tab Когустова Влада Васильевна\\\tab Угольникова Екатерина Алексеевна} \\[1cm]
			{\tab Проверили:\\\tab доцент кафедры ТСиВС\\\tab Дроздова Вера Геннадьевна\\\tab ведущий инженер кафедры ТСиВС\\\tab Ахпашев Руслан Владимирович}
		\end{flushleft}
		\vfill
		{Новосибирск 2020}
	\end{center}
	
\end{titlepage}   

%------------------------------------------------------------

\clearpage

\tableofcontents
\clearpage
\section{Задание курсовой работы}

\tab Требуется разработать web-страницу, отображающую карту местности. На карте необходимо случайным образом отрисовать базовые станции (BS) и абонентские устройства (UE). Базовые станции статичны. Абонентские устройства могут двигаться, могут стоять на месте.

План курсовой работы:
\begin{enumerate}
	\item Создать Web-страницу c картой (yandex || google || openstreetmap).

	\item Создать несколько абонентов (минимум 10) и отобразить их на карте (маркеры\textbackslash картинки\textbackslash схема).

	\item Абоненты должены двигаться только по дорогам (изменение координат абонентов относительно времени).

	\item Шаг изменения координат зависит от скорости абонента.

	\item 50\% абонентов должны ходить со скоростью от 3 до 7 км/ч, 50\% абонентов со скоростью от 30 до 70 км/ч.
\end{enumerate}


\section{Цель работы}

\tab Целью данной курсовой работы является знакомство с языком программирования javascript и использованием API для работы с разными картами (Google, Yandex). Реализация моделирования движения абонентов по карте, при ограничении области движения дорогами. 

\clearpage
\section{Ход работы}
\subsection{Краткое описание алгоритма выполнения курсовой работы}

\tab Для начала мы инициализировали карту и разместили на ней константное количество базовых станций, которые представляют собой марекеры на карте с определнными координатами в виде широты и долготы, а также кастомизированы с помощью изображения из интернета.

Следующим шагом было создать необходимое количетсво абонентов со случайными координатами (они так же являются маркерами) и научить их ходить случайным образом по всей карте.

Потом мы добавили некоторые элементы управления и приступили к работе с API для получения необходимых данных с помощью get-запросов к сервисам Google и Yandex. 

Мы 



\clearpage


\section{Результаты работы}
\begin{figure}[h!]
	\centering
	\includegraphics[width=1\linewidth]{C:/Users/gieko/Desktop/МРС/Л3/images/result1}
	\caption{Фрагмент заполненной базы данных по теме IP-основы и  IP-адресация}
	\label{fig:mpr}
\end{figure}
\begin{figure}[h!]
\centering
\includegraphics[width=1\linewidth]{C:/Users/gieko/Desktop/МРС/Л3/images/results}
\caption{Общая статистика пройденных игр (с учетом участия до использования бота)}
\label{fig:mpr}
\end{figure}
\begin{figure}[h!]
	\centering
	\includegraphics[width=1\linewidth]{C:/Users/gieko/Desktop/МРС/Л3/images/result2}
	\caption{Демонстрация результата работы бота в викторине}
	\label{fig:mpr}
\end{figure}

\clearpage
\section{Вывод}

\tab В результате выполнения курсовой работы создана web-страница для взаимодействия с картами Google Maps. Реализовано моделирование движения абонентов по карте, при ограничении области движения дорогами.

Получены навыки работы с языком программирования javascript и стеком языков разметки html/css; навыки работы с картами и их API, навыки написания get-запросов и обработки ответов, представленных данными в JSON формате. 

В результате взаимодействия с сервисами предоставления API у разных вендоров (Gooble, Yamdex) получен незабываемый опыт несправедливого оценивания количества отправленных get-запросов со стороны Yamdex Maps API, впервые так сильно хотелось дождаться наступления нового дня по Московскому времени. А благодаря Gooble Maps API получен первый в жизни кредит, на целых 300\$!

\section*{Приложение 1}
\addcontentsline{toc}{section}{Приложение 1}
\begin{enumerate}
	\item \href {https://github.com/SLADKAY-KISA/JS_MRS.git}{https://github.com/SLADKAY-KISA/JS\_MRS.git}.
\end{enumerate}
\end{onehalfspacing}

\clearpage
\section*{Приложение 2}
\addcontentsline{toc}{section}{Приложение 2}



\end{document}
